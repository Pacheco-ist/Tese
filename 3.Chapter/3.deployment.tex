\section{Deployment environment} 
\label{sec:deployment} 
 
 
In order to understand how a system based on this generic architecture, Figure \ref{fig:generic}, has to be structured, it is relevant to analyse the system component's description and requirements that would allow it to be deployed. If one starts by analysing the beacons component, one can describe them as the reference points of the system which are in charge of providing nearby users with information relative to their surroundings. These beacons are available to the users and once contacted should make available to the smartphone the address of the its associated location server. Since the beacon only communicates with the smartphone, it isn't required to have any other communication capacity other than that specific to its technology. The last requirement is that the beacon needs to be uniquely identified and have the address of its associated server, while being capable of transmitting that same information to the smartphone. 
 
 
The smartphone service is the component that is in charge of communicating with both the beacons and the location server. This service is a software component of the smartphone application which is available to being called when a location is requested. Once a location is requested by the application, the service initiates communication with nearby beacons through one of its sensors. A service is associated with one and only one technology, as such if the application is to support more than one technology, multiple services should be implemented. This condition is crucial since the required sensor depends on the chosen technology and because it allows the service to quickly tag the data with its associated technology, information which must be delivered to the location server. Once the beacon data has been collected, the same data is forwarded to the beacon's associated server, together with its tag. Upon obtaining a location from the server, the service should forward the data to the application, which is responsible of later communicating with the map server. In terms of requirements, the service needs to have Wi-Fi or at least mobile network available for communicating with both the location and maps server, and access to the sensor required to communicate with the nearby beacon. In the case of BLE it would be the Bluetooth antenna while for QR it would be the camera. 
 
 
The Location server should have a database of all the devices associated to it, each with its exact location of the map. This location is dependent on the method for location description chosen. The server should also be able to handle input from any of the supported technologies and apply the implemented algorithm, so that the user's location is obtained. From this description one can define the requirements of the location server as having in storage the beacons that are associated it as well as their location, being capable of communicating with the smartphone application and being in accordance with the map server, achieved through providing location description that is the same as that present in the maps server. 
 
 
The Map server is responsible for providing the application with the maps that it has requested. The only decision concerning this server is the type of location description that is to be used and consequently the only requirement is that it has to be in accordance with the information on the location server. 
 