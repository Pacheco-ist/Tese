\section{System Requirements} 
\label{sec:requirements} 
 
 
Before defining a generic system it is fundamental to define the requirements that those need to implement. In a system like GEFILOC, these requirements can be of many different types: energetic, functional or scalability requirements. In the end of this document an evaluation of these requirements will be presented. 
 
 
\begin{description} 
\item [Functional requirements] The whole architecture revolves around having a smartphone as the central point for communication. As such, a system like GEFILOC is required to be \textbf{capable of calculating a position}. Such a requirement is dependent on other functionalities. Calculating a position is only achievable if the system is \textbf{capable of gathering information from beacons}, which is also dependent on the system's capability of \textbf{communicating with its chosen technology's beacons}. The other required functionality is that the system needs to be able to display the user's calculated position on a map.  
 
 
 
 
\item [Energetic Requirements] One of the concerns when making use of the smartphone as the central node of communication is the fact that the whole system is reliant on the energetic availability of the smartphone. As such it is crucial to make sure that the energetic impact of the system, when it is functioning, is the smallest possible. 
 
 
The energetic impact per single operation associated to the system depends on multiple factors such as: 
\begin{itemize} 
\item  Energetic capacity of the smartphone, whose values have been increasing over the years in order to support the higher energetic requirements. 
\item Cost of the communication between beacon and smartphone, which depends on the used technology. 
\item Cost of the communication between smartphone and servers, which depends on the size of the messages and the frequency at which they are sent. 
\item Cost of the technology's scanning procedure. 
\item Cost of displaying the maps on the smartphone application. 
\item the overall cost of the application utilisation. 
\end{itemize} 
 
 
When analysing the costs of each of the previously mentioned factors the costs associated to scanning and communication between beacon and smartphone are the most pertinent ones. These cost are dependent on the technology's specific sensor. The remaining communication cost are relatively constant and are independent of the chosen technology. These costs depend mostly on the size of the transmitted messages, which for the ones between smartphone and location server depends on the number of surrounding beacons and for the ones between smartphone and maps server depends on the size of the maps on the maps server.   
 
 
The overall cost of the system is dependent on the frequency of the operation. The system's energetic impact can be reduced through the optimisation of the system's update frequency. This optimisation should target the user's movement, since while the user is stationary, no updates on its position are required. 
Update frequency also has an impact on the precision of the calculated position. This situation introduces a tradeoff on this frequency between consumption and accuracy. While a high update frequency allows the system to provide to the user precise information about its location, it also increases the system's overall consumption. 
 
 
\item [Interoperability requirements] The concept behind the proposal of a generic architecture framework is that, besides multiple systems, which use different technologies, being capable of functioning under the same architecture and interface, they can also work together seamlessly.  
 
 
In order to support interoperability, i.e. a building location system attempting to use multiple supported technologies, it needs to be capable of including new supported technologies onto its implementation. In order to do so, there are three aspects that need to be fulfilled by the system: 
 
 
\begin{itemize} 
\item It has to be capable of functioning with two technologies at the same time. This involves being capable of interacting with beacons from the two technologies. As such, the system has to be capable of communicating with beacons and allow the extraction of their information, for either technology. At the same time, the location server needs to be capable of calculating the user's position independently of the data's type. 
 
 
\item It has to be capable of working with two location calculation algorithms. This aspect enforces the location server to be capable of having two implemented algorithm, both capable of producing a location from the same provided data.  
 
 
\item It has to capable of working with two location description formats. Such condition affects both servers. The location server is allowed to provide locations in one of two different types of formats, while the maps server is capable of providing a map in either. 
The smartphone needs to be able to display to the user its location, no matter the used format. 
\end{itemize} 
 
 
 
 
\item [Environmental requisites] The environmental requirements are independent from the system's technology. GEFILOC's architecture presents beacons as the components responsible of providing the user with information about its surroundings. As such, two beacon requirements emerge: 
\begin{itemize} 
\item Since beacons are only required to communicate with the user's smartphone, which is accomplished through its technology's protocol, no network access is required. 
 
 
\item Since beacons are deployed onto the environment, a crucial requirement is their capability of working without need for maintenance for long periods of time. As such, their energy consumption should be the lowest possible. If such requirement isn't met, the system's maintenance cost increase.  
\end{itemize} 
 
 
\end{description} 
 