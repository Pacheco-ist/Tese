\begin{resumo}

Atualmente existe uma grande procura por sistemas de localização. Apesar do sistema de Global positioning system (GPS) conseguir obter ótimos resultados em ambientes exteriores, o mesmo não acontece quando aplicado a ambientes interiores. Assim sendo, a investigação no campo de sistemas de localização indoor intensificou-se com o objetivo de produzir resultados ao nível do GPS. Esta tese investiga o estado atual dos sistemas de localização indoor, através do estudo das tecnologias e técnicas para calcular localizações existentes. De seguida é realizada uma analise através de uma visão moderna, focalizada em serviços de localização através de smartphones. Devido ao grande número de sistemas existentes, é sugerida uma arquitetura genérica capaz de ser aplicada a sistemas que tencionam usar smartphones como peça central. Esta arquitetura, nomeada GEFILOC, surge com o objetivo de permitir interoperabilidade através do isolamento das componentes principais destes tipos de sistemas. Outro aspeto relevante, é a capacidade de fácil inclusão de novas tecnologias ou algoritmos na estrutura base.  
De modo a testar o GEFILOC, foi implementado um sistema baseado em Bluetooth de baixa energia (BLE). Este sistema permitiu a análise do seu consumo energético, tendo sido possível concluir que o impacto associado à comunicação internet era baixo. Esta analise era fundamental, visto que este seria o único custo introduzido pela utilização do GEFILOC. Este custo está associado ao isolamento das diversas componentes em servidores e assim foi possível provar que o compromisso é atingido sem causar um impacto relevante.     

\end{resumo}