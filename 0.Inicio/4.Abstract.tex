\begin{abstract} 
 
Nowadays there is a great demand for positioning systems. While GPS' performance is amazing for obtaining outdoor locations, the same doesn't happen in indoor environments. As such, indoor positioning systems have been studied in order to achieve similar results. In this thesis, one studies the actual state of indoor systems through an analysis on the usable technologies and techniques. A critical analysis is conducted through a modern perspective centered on location services through smartphones. Since there is a great number of different indoor location systems, one found a need to create a framework capable of hosting such systems in the form of a smartphone-centric architecture. This generic architecture attempted to achieve  interoperability  through the isolation of the main components of indoor systems. As such GEFILOC, a generic framework for indoor location systems, was created. GEFILOC would allow for systems that made use of it, to be capable of operating together on the same device without additional requirements. In order to test the proposed framework, a Bluetooth low energy-based system was implemented. This system allowed the analysis of the energetic costs of the architecture, having shown that the energy consumption associated to network communication had a low impact. This observation was fundamental since this cost was introduced by the isolation of components onto servers, and analysis proves that the proposed trade off comes at an acceptable cost.   
 
\end{abstract} 
 