\section{Bluetooth Low Energy}
\label{sec:BLE}

Bluetooth is a wireless technology that was created in 1994 with the objective of replacing cables connecting fixed or portable devices. At this point in time Bluetooth Special Interest Group is in charge of developing and managing this technology characterized by its robustness, low energy consuption and low cost. 

The \ac{BLE} protocol was introduced with the Bluetooth Core Specification version 4 (also called Bluetooth Smart) circa 2010 alongside two other protocols.  Out of the three, \ac{BLE} standed out for its lower power consuption, lower complexity and lower cost, while allowing for  device discovery, connection establishment and connection mechanisms. Due to its characteristics, the \ac{BLE} protocol was utilized in various \ac{IoT} applications.  

\subsection{Subsection A}
\label{subsec:subasectionA}

This would be a citation \cite{dummy}.

\ac{acro} 
% The first time you use this, the acronym will be written in full with the acronym in parentheses: supernova (SN). At later times it will just print the acronym: SN.

\acf{acro}
% written out form with acronym in parentheses, irrespective of previous use

\acs{acro}
% acronym form, irrespective of previous use

\acl{acro}
% written out form without following acronym

\acp{acro}
% plural form of acronym by adding an s. \acfp. \acsp, \aclp work as well.

As seen in \cite{wiki}. \emph{Enfatizar}

\subsection{Subsection B}
\label{subsec:subbsectiona}

\begin{figure}[H]
	\centering
		\includegraphics[width=0.5\linewidth]{2.Chapter/dummy.pdf}
	\caption[Dummy Figure Caption for List of Figures.]{Dummy Figure Caption.}
	\label{fig:dummyfigure1}
\end{figure}

Remember you can change the reference style. Another dummy citation \cite{site}.