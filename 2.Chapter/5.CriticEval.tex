\section{Related work}
\label{sec:related}

Pegar em tudo e comentar:
	usabilidade nos tempos de hoje
	integração com smartphones


RFID:
	\ac{RFID} in the current days isn't much of technology for indoor location systems. This is caused by multiple factors such as its incompatibility with smartphone, which by itself is already a critical factor for not being capable of being used in indoor systems other than the ones for research purposes. One recent system using \ac{RFID} was presented by Han Zou et al. \cite{rfid_sys} which makes use RFID tags and sensors, a \ac{RFID} reader and a location server. The general procedure for computing a location is that the tags emited its ID which is caught by the sensors, the sensors make use of its external power supply to allow the creation of a continuous wireless connection with the reader. Once the reader receives the data he forwards it to the server where the location is obtained. This system introces one of the issues of RFID which is the high cost of readers. This situation creates a big constraint on the architecture since making the mobile user the central piece of the architecture would require a \ac{RFID} reader per person. As such nowadays \ac{RFID} isn't a technology used for indoor location but instead for people or object tracking, since this approach only requires the person/object to have a unique tag. 
	
	Landmarc e spoton????

WLAN:
	WLAN is a technology that was made for positioning but since it contains \ac{RSSI} information it is possible to use it with that purpose. One of its strongest points is its availability, as it's hugely wide spread around the world and it doesn't require any extra infrastructure, other than the already existing one, to function. As such nowadays this technology can be useful when attempting to create an indoor location system capable of being utilized by smartphones in any environment. This is only possible since pretty much every single smartphone is equiped with a WLAN antenna.
	WLAN \ac{RSSI}-based indoor location systems can be of two types, trilateration or fingerprint.

	When imagining a scenario for a system implemented in multiple sites, the fingerprint technique raises questions which can be analysed by looking into the RADAR system. 

	Radar's architecture defines the mobile user as the broadcaster for the whole of its process, be it offline and online phases of the fingerprint method as described in (xXXXX).Despite the choice, P. Bahl et al,(cite radar) comment on their choice consequences, imagining that in a real situation, since the number of mobile users is vastly bigger than that of base stations, the inversion of the roles would be beneficial in order to decrease the complexity and the workload. Passing the broadcast role to the base stations while having the mobile users are listeners would make it so that for a certain floor the complexity would be constant, i.e. equal to the number of stations, instead of being unstable as the number of current mobile users. This change of architecture would allow for much more stable system, while requiring that the mobile users to perform an extra action in the form of a scan, a scan for capture of base station's broadcasts. Radar was design in a way that the collected information by the base stations were forwarded to a central computer, where they were processed in order to obtain a user's location. As such any existing user on the network was capable of being tracked but the information wasn't made available to the mobile user. Device deployment in this system isn't an issue as Wi-Fi is massive deployed technology present in most existant devices. At the time of development this system made use of computer with wi-fi available, as such no system specidic devices were required at least for the online phase. Due to the choice of fingerprint as location method, aditional measures conserning data collection are always required.

	By analysing RADAR its possible to understand the contraints of the fingerprint technique. As such only the trilateration technology could be of use in a large scale environment. With the idea of keeping the smartphone as the central piece of the  architecture, it would be on him to capture the \ac{RSSI} signals from the nearby access points and forward that same information to their associated servers for the rest of the location procedure.


	Due to the WLAN characteristics and its worldwide availability, it is often used in hybrid indoor location systems. 



Infrared: 
	Most recent stuff is LE based at least for positioning, rest is most for tracking
	Recent work using leds bulbs and smartphones
	http://panhu.me/pdf/Epsilon.pdf

	The badge system's architecture is considerably diferent from the one proposed in this work as the role of the beacon(sensor receivers) and the smartphone (badge devices) are exchanged. This exchange in roles allows to reduce the energy cost on the mobile device since it isn't up to it to communicate with the server and due to the design of the system the energetic balance between scanning and performing a signal broadcast for a tenth of a second each fifteen seconds isn't of relevance. Another aspect of this architecture is the fact that the collected data is sent to the central station, the only place where it will be available, making it only capable of tracking individual users and not providing the users with their location.Since the required building maps are already in the central station there is no need a maps provider. In terms of device distribution, this system makes use of system specific devices, which are already set up and inserted into the system and are distributed to anyone or anything that is to be tracked.

























Roles. (existance) consequences
Energy
Device difusion

structures:
badge-      Badge (phone) sends signal
			Sensors (receivers) capture signal and forward to Master Station
			Master statio (centralized system) computes location and makes it available on the central computer




bat- 		Central station sends a signal indicating which bat to locate (message to all "area" controllers)
			Controllers forward message through radio to nearby sensors and bats
			Bat sends US signal upon receival of message
			Sensors reset upon receival and wait for US signal
			Sensors receive signal and send time of flight to central
			central computes location and makes it available on central computer

The bat system's architecture is very similar to the one deployed by the badge system, with the big difference being that its location is tracked upon request instead of being periodic. In energetic terms, it is also similar with the increased cost of having an additional operation, listening for central station's messages. Bat deployment is also manually done, with each person or object required to carry a system specific device. Each user's position is tracked and not provided to the user, while being made available through the maps provider extant within the central station.


Radar-		Offline Phase:
				Mobile Host broadcasts at marked points for fingerprint
				Proposed to inverse the roles
			Online Phase:
				Same thing




cricket- 	Beacons send RF broadcast periodically and at the same time a US signal
			Mobilehost start counting time when they receive a US signal and stop when the RF arrives
			Mobile Host computes the distance

Add to cricket info on this paper - http://www.cse.wustl.edu/~lu/cse521s/Papers/cricket.pdf


The cricket system is completely different from the other three systems already presented, in terms of architecture. Cricket's Architecture is completely decentralized and the beacons and mobile users have the same roles as the proposed generic architecture of this work. Cricket's beacons broadcast periodically RF and US signals while the mobile user is in charge of listening for both of the signals. Data collection happens entirely on the mobile user's side, as explained on section( xXXXx), in the form of \ac{TDoA}. Location computation is obtained directly on the cricket's listeners and this information is made available through this device's API. As such the mobile user's device can inquire the listeners about its position. This design , when compared to the the presented generic framework, reduces a communication step, the one between device and location server. This situation allows the user to keep its privacy intact as no one else has access to the its location. On the other hand, location computational work is passed onto the mobile user, which can make an impact on the performance depending on the algorithms complexity. Another issue is the fact the code deployed on mobile devices can cause issues if needed to be update, creating a versioning problem that can be solved by having the required code in a centralized station. Cricket's system was created without providing a map based location although as stated by the authors (Add cite to cricket paper), with the deployment of a map server, as in the presented generic framework, by sending the calculated location to this map provider, a location system could be easily provided to the user.


zonith- 	beacon on the Wall/Ceiling collects periodically all nearby ble devices
			beacons forward information to central, making it available
			Workers and temporary entities (visiting people) need to be manually inserted into the database

Zonith's architecture was created solely for tracking of people and as such presents inverted roles for beacons and mobile users when compared to the generic framework. The system requires users that need to be tracked to manually insert their device's information, be it system specific or the user's personal device, onto the system. The carried devices should be always in advertising mode so that the existant listeners can capture their signal. This design although similar to the imposed on the mobile devices of the badge or bat systems, since its an always active feature should have a much bigger impact on the energy requirements. The beacon listeners forward their gathered data onto the central station, where all the data of the users to be tracked are kept and the system's functionalities are computed. 

They have a indoor location feature for rescuing teams, showing a map with the route to the user in danger's location. 


ibeacon- 	ibeacon sends proximity packets
			mobile devices capture them and forward to server
			server processes back to the mobile user

The ibeacon hospital system's architecture is very similar to the presented generic framework. There are beacons which broadcast signals containing location specific information, which are to be capture by nearby mobile user's devices. Upon receiving the beacon's information, they forward it into the central server which analyzes the location and gives back to the user information based on it. The maps required for displaying the user's location were deployed on the user's application upon installation of the service. Upon receiving the user's location information back from the server, it is displayed on the local accessible maps. By removing the existance of a map provider server, the scalability of the system is drastically reduced. While it is acceptable for closed environment's service, i.e. app designed for a specific location such as an hospital, when making an indoor location system that is to be used in multiple places, a general exterior map provider server makes it so that a smartphone application doesn't require to have locally all the required maps.