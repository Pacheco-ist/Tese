\section{Related work}
\label{sec:related}

Pegar em tudo e comentar:
	usabilidade nos tempos de hoje
	integração com smartphones


RFID:
	\acf{RFID} in the current days isn't much of technology for indoor location systems. This is caused by multiple factors such as its incompatibility with smartphone, which by itself is already a critical factor for not being capable of being used in indoor systems other than the ones for research purposes. One recent system using \ac{RFID} was presented by Han Zou et al. \cite{rfid_sys} which makes use RFID tags and sensors, a \ac{RFID} reader and a location server. The general procedure for computing a location is that the tags emited its ID which is caught by the sensors, the sensors make use of its external power supply to allow the creation of a continuous wireless connection with the reader. Once the reader receives the data he forwards it to the server where the location is obtained. This system introces one of the issues of RFID which is the high cost of readers. This situation creates a big constraint on the architecture since making the mobile user the central piece of the architecture would require a \ac{RFID} reader per person. As such nowadays \ac{RFID} isn't a technology used for indoor location but instead for people or object tracking, since this approach only requires the person/object to have a unique tag. 
	

WLAN:
	WLAN is a technology that was made for positioning but since it contains \acf{RSSI} information it is possible to use it with that purpose. One of its strongest points is its availability, as it's hugely wide spread around the world and it doesn't require any extra infrastructure, other than the already existing one, to function. As such nowadays this technology can be useful when attempting to create an indoor location system capable of being utilized by smartphones in any environment. This is only possible since pretty much every single smartphone is equiped with a WLAN antenna.
	WLAN \ac{RSSI}-based indoor location systems can be of two types, trilateration or fingerprint.

	When imagining a scenario for a system implemented in multiple sites, the fingerprint technique raises questions which can be analysed by looking into the RADAR system. 

	Radar's architecture defines the mobile user as the broadcaster for the whole of its process, be it offline and online phases of the fingerprint method as described in (xXXXX).Despite the choice, P. Bahl et al,(cite radar) comment on their choice consequences, imagining that in a real situation, since the number of mobile users is vastly bigger than that of base stations, the inversion of the roles would be beneficial in order to decrease the complexity and the workload. Passing the broadcast role to the base stations while having the mobile users are listeners would make it so that for a certain floor the complexity would be constant, i.e. equal to the number of stations, instead of being unstable as the number of current mobile users. This change of architecture would allow for much more stable system, while requiring that the mobile users to perform an extra action in the form of a scan, a scan for capture of base station's broadcasts. Radar was design in a way that the collected information by the base stations were forwarded to a central computer, where they were processed in order to obtain a user's location. As such any existing user on the network was capable of being tracked but the information wasn't made available to the mobile user. Device deployment in this system isn't an issue as Wi-Fi is massive deployed technology present in most existant devices. At the time of development this system made use of computer with wi-fi available, as such no system specidic devices were required at least for the online phase. Due to the choice of fingerprint as location method, aditional measures conserning data collection are always required.

	By analysing RADAR its possible to understand the contraints of the fingerprint technique. As such only the trilateration technology could be of use in a large scale environment. With the idea of keeping the smartphone as the central piece of the  architecture, it would be on him to capture the \ac{RSSI} signals from the nearby access points and forward that same information to their associated servers for the rest of the location procedure.


	Due to the WLAN characteristics and its worldwide availability, it is often used in hybrid indoor location systems. 



Infrared: 

	In order to analyze the \acf{IR} technology one can initiate by studying the badge system, which was described in section \ref{subsec:badge}.

	The badge system's architecture is considerably diferent from the one proposed in this work as the role of the beacon(sensor receivers) and the smartphone (badge devices) are exchanged. This exchange in roles allows to reduce the energy cost on the mobile device since it isn't up to it to communicate with the server and due to the design of the system the energetic balance between scanning and performing a signal broadcast for a tenth of a second each fifteen seconds isn't of relevance. Another aspect of this architecture is the fact that the collected data is sent to the central station, the only place where it will be available, making it only capable of tracking individual users and not providing the users with their location.Since the required building maps are already in the central station there is no need a maps provider. In terms of device distribution, this system makes use of system specific devices, which are already set up and inserted into the system and are distributed to anyone or anything that is to be tracked.

	Due to badge's age, it can't be relied on when making a judgement on the usability of \ac{IR} in the present time. As such it is of relevance to look at recent works dependent on this technology such as the Epsilon project \cite{epsilon}, which makes use of Light-emitting Diode (LED) and light sensors.  

	Epsilon makes use of LEDs and their capacity to both illuminate, the main reason for it's utilized and wide available il multiple indoor environments, and their capacity to communicate. By making use of the LED's ability to instantaneously change between on and off, it is possible to carry digital data in the visible light carrier through Pulse Width Modulation (PWM). In this system each of the used LEDs was self-contain, i.e. it knew its own location, which was insert into the data transmitted to the mobile user. The mobile user would capture the LED information through its light sensor, and consequently decode the received data from each of the available sources in order to make use of a trilateration system to obtain its position.

	Epsilon presents a solution which is easy to use in today's world due to the availability of LED lighting in indoor environments, with the only requirement being the adaption of the existing infrastruture to the required by the system. In order for the system to work, it required the mobile user to carry a device with an incorporated light sensor, which is the case of pretty much any existant smartphone.
	Epsilon's architecture was created with the intention of having a "plug-and-play" kind of system, made possible through the self-contained LEDs and location calculation made on the smartphone. However in order to be used in a broader environment it could be adaptable to this work's proposed architecture.

	Much like the authors already sugest in their work, instead of the self-contained LEDs, they could use a back-end service for mapping the IDs of each LED to their location, by relying on a network connection. The mentioned service would be the equivalent of the location server component on the proposed architecture, with the additional possibility of passing the location computation (which was implemented to occur on the smartphone) to the location service. Since no map component is mention on the work, there wouldn't be any conflict with the addition of the map server, which would rely on the already needed network connection.


Ultra-Wideband:

	\acf{UWB} is a technology that isn't very practical for indoor location in today's world. The main reason for it is the incompatibility with the existant modern day smartphones.
	Existant \ac{UWB} system are dependent on system specific hardware which invalidades the possibility of having a wide scale indoor location systems.

	When analyzing recent \ac{UWB} system, the most common ones are tracking only systems. These systems are found useful in construction environments, where tracking crane's position is of relevance \ref{uwb-ex1}. This position tracking is accomplished by setting up a network of \ac{UWB} sensors on the construction site and placing \ac{UWB} tags on the crane or parts of the crane, if the whole position and pose of the vehicle are of relevant, that is to be tracked. If additional objects were to be tracked, it would only be required to add tags to the object in question.

	In terms of viability for indoor location, there are commercially available solutions such as the one presented by Decawave \ref{uwb-ex2}. Decawave presents a \ac{UWB} transciever in the form of a hardware module. The proposed module makes use of \acf{RToF} to obtain distance between itself and another module of the same kind. This transciever can be used to implement indoor location systems based on \ac{UWB} although it won't be compatible with existant smartphones.


Optical:

	As described in section \ref{subsec:others}, optical systems are systems which are dependent on camera as means to obtain input for indoor location. These kind of system are easily usable in today's world due to all the technological improvements of cameras and their integration into every single existant smartphone. Although it was described in section \ref{sec:indoortech} as a single technology, nowadays smartphone cameras as used as input receivers in \ac{IR} systems. One example is the already analyzed system, Epsilon, where the camera is used as a mean to capture \ac{IR} signals from LEDs. 

	Other variants of camera-based systems make use of depth cameras with the intention of build dense 3D maps of indoor environments \ref{camera_ex}, which can be useful in robot navigation. Due to their high cost and incompatibility with modern day smartphone, these variants can't be used for large scale indoor location systems.


FM: 
	The \ac{FM} radio technology can be seen as an alternative to WLAN, due to their similarities. Much like WLAN, \ac{FM} recievers are present in pretty much every single existant smartphone which removes the smartphone component from the list of possible contrainsts. FM as a technology for indoor location can be used, but often times is discarted in favor of other possibilities. When compared to WLAN, while it has some advantages such as lesser power consumption on smartphones, its bigger range and power, or even the fact that its less prone to interference from human presence or objects due to its bigger frequency band, when compared to WLAN already existant infrastructure and availability, all the positive sides become less meaningful. 

	Sungro Yoon et al. \cite{fm_ex}, proposed an \ac{FM} system capable of providing indoor location through a fingerprint method and the signals commercial \ac{FM} radio stations. Since it's a fingerprint-based system, the whole process and its pro's and con's have already been heavily commented on. The big addition with this system is the utilization of a publically available signal,i.e. existant commercial radio stations, which means that there is no need for extra infrastructure. 
	When analyzing the possibility of adapting this system onto the proposed generic architecture, i.e. attempting to make use of this work in order to idealize a higher scale system, one can say that it is feasable. Since the smartphone functions as a receiver, with the radio towers, which include their information,location and power, functioning as the "beacons", the missing components would be making the fingerprint database on a centralized server, with the same happening to the maps provider.
 

Zigbee:

	Zigbee is a technology that has been gaining traction due to its low power characteristics, which are very popular in \ac{IoT}, but as of today hasn't been incorporated into smartphones. As such a smartphone reliant system using zigbee isn't possible only by itself, it would required the addition of a zigbee dongle to allow smartphones to communicate with other zigbee devices. With such restraints, zigbee hasn't been very developed in the indoor location field.
	
	The few existing zigbee system make use of the fingerprint method for data collection and location computation, such as the example provided in section \ref{subsec:others}. With the development of IoT it is possible that one day smartphone will be capable of communicating with other zigbee devices, allow for further development on indoor location.


Ultrasonic:

	In order to analyze \ac{US} it is of relevance to review two systems, the bat and cricket systems presented on section \ref{sec:related}, that make use of this technology and analyze how they could be adapted to todays reality and the generic smartphone centric architecture for an wide scale indoor location system.

	The bat system's architecture is very similar to the one deployed by the badge system, with the big difference being that its location is tracked upon request instead of being periodic. In energetic terms, it is also similar with the increased cost of having an additional operation, listening for central station's messages. Bat deployment is also manually done, with each person or object required to carry a system specific device. Each user's position is tracked and not provided to the user, while being made available through the maps provider extant within the central station.

	The cricket system is completely different from the other remaining systems already presented, in terms of architecture. Cricket's Architecture is completely decentralized and the beacons and mobile users have the same roles as the proposed generic architecture of this work. Cricket's beacons broadcast periodically RF and US signals while the mobile user is in charge of listening for both of the signals. Data collection happens entirely on the mobile user's side, as explained on section \ref{sec:related}, in the form of \ac{TDoA}. Location computation is obtained directly on the cricket's listeners and this information is made available through this device's API. As such the mobile user's device can inquire the listeners about its position. This design , when compared to the the presented generic framework, reduces a communication step, the one between device and location server. This situation allows the user to keep its privacy intact as no one else has access to the its location. On the other hand, location computational work is passed onto the mobile user, which can make an impact on the performance depending on the algorithms complexity. Another issue is the fact the code deployed on mobile devices can cause issues if needed to be update, creating a versioning problem that can be solved by having the required code in a centralized station. Cricket's system was created without providing a map based location although as stated by the authors (Add cite to cricket paper), with the deployment of a map server, as in the presented generic framework, by sending the calculated location to this map provider, a location system could be easily provided to the user.

	In order to look at \ac{US} system's capabilities in today's world one can look into the work of Viacheslav F. et al \cite{us_ex}. Their work presented a location system based on the capability of producing \ac{US} signals using a smartphone's microphone. The presented system made use of a smartphone as the central piece, which interacted with the system through a mobile application. Altough it was a promising approach on indoor location, it falls under the limitations of the technology. Since \ac{US} signals require an existing \ac{LOS} between sensor and smartphone, the infrastructural requirements are higher than many of the remaining technologies.























Roles. (existance) consequences
Energy
Device difusion

structures:
badge-      Badge (phone) sends signal
			Sensors (receivers) capture signal and forward to Master Station
			Master statio (centralized system) computes location and makes it available on the central computer




bat- 		Central station sends a signal indicating which bat to locate (message to all "area" controllers)
			Controllers forward message through radio to nearby sensors and bats
			Bat sends US signal upon receival of message
			Sensors reset upon receival and wait for US signal
			Sensors receive signal and send time of flight to central
			central computes location and makes it available on central computer



Radar-		Offline Phase:
				Mobile Host broadcasts at marked points for fingerprint
				Proposed to inverse the roles
			Online Phase:
				Same thing




cricket- 	Beacons send RF broadcast periodically and at the same time a US signal
			Mobilehost start counting time when they receive a US signal and stop when the RF arrives
			Mobile Host computes the distance

Add to cricket info on this paper - http://www.cse.wustl.edu/~lu/cse521s/Papers/cricket.pdf





zonith- 	beacon on the Wall/Ceiling collects periodically all nearby ble devices
			beacons forward information to central, making it available
			Workers and temporary entities (visiting people) need to be manually inserted into the database

Zonith's architecture was created solely for tracking of people and as such presents inverted roles for beacons and mobile users when compared to the generic framework. The system requires users that need to be tracked to manually insert their device's information, be it system specific or the user's personal device, onto the system. The carried devices should be always in advertising mode so that the existant listeners can capture their signal. This design although similar to the imposed on the mobile devices of the badge or bat systems, since its an always active feature should have a much bigger impact on the energy requirements. The beacon listeners forward their gathered data onto the central station, where all the data of the users to be tracked are kept and the system's functionalities are computed. 

They have a indoor location feature for rescuing teams, showing a map with the route to the user in danger's location. 


ibeacon- 	ibeacon sends proximity packets
			mobile devices capture them and forward to server
			server processes back to the mobile user

The ibeacon hospital system's architecture is very similar to the presented generic framework. There are beacons which broadcast signals containing location specific information, which are to be capture by nearby mobile user's devices. Upon receiving the beacon's information, they forward it into the central server which analyzes the location and gives back to the user information based on it. The maps required for displaying the user's location were deployed on the user's application upon installation of the service. Upon receiving the user's location information back from the server, it is displayed on the local accessible maps. By removing the existance of a map provider server, the scalability of the system is drastically reduced. While it is acceptable for closed environment's service, i.e. app designed for a specific location such as an hospital, when making an indoor location system that is to be used in multiple places, a general exterior map provider server makes it so that a smartphone application doesn't require to have locally all the required maps.