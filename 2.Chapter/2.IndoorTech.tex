\section{Indoor location support technologies}  
\label{sec:indoortech}  
  
  
When looking at the state of indoor positioning systems, it's clear that there isn't one technology that is better than all of the others. Therefore, it's important to look at each of the possible technologies individually and assess its benefits and drawbacks as well as their performance\cite{survey1}.  
  
  
In this chapter, many existent indoor positioning technologies are analysed. The most pertinent ones, \acf{RFID}, Commodity wireless technologies, Infrared, Ultra-wide band and Bluetooth low energy, are explained in a more detailed manner, while the less utilised technologies are described in Subsection \ref{subsec:others}. For each of the present technologies a description is provided about their nature, tags and pros and cons, all of which is complemented with at least one existent system that makes use of the specific technology being described.   
  
  
The information provided on each technology was gathered from a set of surveys on indoor location \cite{surveythesis,survey2,survey1}, as well as the information present on the mentioned systems associated to each.  
  
  
\subsection{RFID}  
\label{subsec:rfid}  
  
  
\ac{RFID} is a technology for storing and retrieving data through electromagnetic transmission to an RF compatible integrated circuit. A \ac{RFID} system is composed by three components: readers, tags and the communication between both. The reader is capable of reading the data that is being emitted from \ac{RFID} tags via radio waves and the data usually consists of the tag's unique identification number which can be related to the tag's available position information in order to obtain the user's position. This communication is achieved by having a well-defined radio frequency and protocol which allows for reading and transmitting data. The \ac{RFID} tags can be of two types: active or passive.  
  
  
Active tags are small transceivers equipped with an internal battery, which makes them heavier and more costly while allowing for longer detection ranges when compared to their counterparts \cite{surveythesis}. These tags are suited for identification of important units moving through rough processes or positioning in system where location estimation is often carried out through fingerprinting on \ac{RSSI}.  
Passive tags are operated without the need of a battery since they are capable of receiving enough energy in the form of radio frequency waves from nearby \ac{RFID} scanners in order to transmit back the answers. These tags are used to replace the barcode technology since they are much lighter, smaller and less expensive than the active tags which allows for a relative inexpensive installation and low maintenance caused by not having batteries. One of its drawbacks is that their range is very limited, circa 2 meters, which demands for higher density of tag deployment.  
  
  
\ac{RFID}'s biggest advantages are the non required \ac{LOS} characteristics, their capability of working at high speeds and their relative low cost\cite{survey2}. The non required \ac{LOS} characteristics comes from its radio frequency nature. Radio frequency signals are composed of electromagnetic waves and as such are capable of passing through obstacles at the cost of signal strength. As such this technology is often used for tracking objects in automobile assembly industry or warehouse management and tracking of people or animals.  
One of its most relevant projects is the SpotON \cite{spoton}, a tagging technology for three dimensional location sensing based on radio signal strength analysis. The tags used are custom devices that operate either standalone or as a plug in card enabling larger devices to take advantage of location-sensing technology. They are low power, small and capable of being accurate while having the computing capacity for relevant tasks such as caching, authentication, among others. SpotON tags utilise the received \ac{RSSI} as a metric for obtaining inter-tag distance.  
  
  
Another important project using \ac{RFID} is LANDMARC\cite{landmarc} which utilises active tags to produce a location sensing system for locating objects inside buildings. Its objective was to demonstrate that active tags can in fact be viable and cost-efficient for indoor location sensing. One of the problems found was that the hardware wasn't capable of providing \acf{RSSI} readings( a value used to evaluate the strength of a radio signal), as such the used readers scan through eight discrete power levels in order to estimate the \ac{RSSI}.   
  
  
\subsection{Commodity wireless technologies}  
\label{subsec:wifi}  
  
  
Commodity wireless technologies can be used to estimate the location of a mobile user that resides inside the network. Nowadays Wi-Fi positioning systems have become the most widespread approach for indoor location systems since \ac{WLAN} access points are readily available in many indoor environments and any Wi-Fi compatible device (smartphones, laptops, tablets) can be located without the need of installing extra software or manipulating the hardware. Its popularity is also due to its range of 100 to 50 meters and since \ac{LOS} isn't required. One issue of \ac{WLAN} signals is that they suffer attenuation from static environment such as walls and movement of furniture and doors. In these kind of systems position computation is obtained through TOA, AOA, RSS, and CSI, which are properly analysed in Section \ref{sec:techniques}, with multiple projects for each one of the existing methods. The most widely used is the \ac{RSSI}, which suffers from severe multipath effects leading to propagation model failures and as such inaccuracy in distance measurement. With these problems in mind a technique called RSSI-based fingerprinting is often used in order to improve performance.  
Most recently an alternative to RSSI has been researched called \ac{CSI}. \ac{CSI} is widely available on commercial products and it represents the channel conditions over individual OFDM subcarriers across the \ac{PHY} layer. One of the improvements is that instead of obtaining one \ac{RSSI} value per packet, multiple \ac{CSI} values can be obtained from multiple subcarriers at a time. FILA \cite{fila} was a project that attempted to use \ac{CSI} for locating targets in complicated indoor environments where RSSI wasn't reliable due to multipath. This system is capable of extracting the \ac{LOS} path for distance calculation through time-domain multipath mitigation and frequency-domain fading compensation and with a simple trilateration calculation they were able to achieve a much better performance than with \ac{RSSI} for these kind of scenarios. The performance comparison showing the differences in temporal stability between \ac{RSSI} and \ac{CSI} can be seen on Figure \ref{fig:fila}.  
  
  
\begin{figure}[H]  
\centering  
\includegraphics[width=0.5\linewidth]{2.Chapter/fila-comparison.png}  
\caption[Comparison of temporal stability (Ref \cite{fila}) ]{Comparison of temporal stability (Ref \cite{fila})}  
\label{fig:fila}  
\end{figure}  
  
  
  
  
\subsection{Bluetooth low energy}  
\label{subsec:ble-tech}  
  
  
Bluetooth low energy was introduced as an improvement to the already existent Bluetooth, aimed at Internet of Things (IoT). Its most relevant improvements from the classic Bluetooth were the reduced power consumption, lower complexity and lower power consumption. \ac{BLE} operates in the same frequency range as the classic Bluetooth, allowing them to make use of the same antenna, and as Wi-Fi. This technology is known for its short-range, overall low power consumption and low-cost transceiver chips.  
\ac{BLE}'s initial target was localised advertising and "near-me" applications, due to its proximity sensing capacity. Nevertheless, its viability as an indoor location technology has been studied, showing that it can be used alongside fingerprinting or with proximity detection. Both techniques make use of \ac{RSSI}, which in the case of \ac{BLE} is not a reliable measure. This condition is caused by the sharing of the same frequency band as Wi-Fi, leading to possible signal interference that causes signal fluctuation. In Section \ref{subsec:ble}, examples of \ac{BLE} systems are provided.  
  
  
\subsection{Infrared}  
\label{subsec:ir}  
  
  
\ac{IR} systems are mostly used for tracking objects or people. \ac{IR} wavelengths are invisible to the human eye under most circumstances, making this technology less intrusive than those which are visible. This technology is widely available in various common devices such as mobile phones, PDA's and TV's and requires \ac{LOS} communication between receiver and transmitter, preferably without interference from strong light sources. One of the most relevant systems based on \ac{IR} is the Active Badge system which is described in Section \ref{sec:related}. There are three methods of exploiting infrared signals: through active beacons, infrared imaging or artificial light sources.  
  
The active beacon's approach is the one employed by the active badge system and it involves placing fixed \ac{IR} beacons on known positions. The density of deployment of beacons has a direct impact on the system's accuracy. If a system was required to achieve room-based accuracy, i.e. being able to tell in which room a user is located, a beacon per room would be sufficient.   
  
Infrared imaging, also known as passive \ac{IR} systems, makes use of sensors operating in the \ac{IR} spectrum which are capable of obtaining a complete image of the surrounding from thermal emissions. This approach doesn’t require the deployment of any extra hardware or tag for determining the temperature of objects or people but it does get compromised in the presence of strong radiation from the sun. Some known equipments that use this approach are thermal cameras, infrared sensors for motion detection or thermocouples used to measure temperature contact free.   
  
\ac{IR} systems based on artificial light sources are a good alternative to the ones that operate on the visible spectrum. A very well known example is the Microsoft Kinect system which uses continuously‐projected infrared structured light to capture 3D scene information with an infrared camera. This system is capable of tracking a person's movement up to 3.5 meters with a precision of a few centimeters.   
  
\subsection{Ultra-Wideband}  
\label{subsec:uwb}  
  
  
\ac{UWB} is a radio technology aimed at short-range high-bandwidth communication. Its best characteristics are the capacity of being resistant to multipath and to some degree being capable of penetrating building materials, such as concrete and wood, with low power consumption. Both these factors allow \ac{UWB} to achieve high positioning accuracy while the latter enables to address the range in non line-of-sight conditions and makes inter-room ranging possible. Being able to penetrate building material creates precision issues due to the increase in data complexity, making data interpretation one of the biggest challenges to be faced. The usual structure of a \ac{UWB} system has a stimulus radio wave generator and receivers which capture the propagated and scattered waves and it has four types of methods for position calculation.  
The first one, passive \ac{UWB}, attempts to track objects or people through signal reflection. This method doesn't require any sort of tag to be carried by the user or attached to the object and requires only at least on emitter and a few listeners to obtain a location. Since the locations of the antennas are known and it is possible to estimate the distance from user to listener through \ac{ToA} or \ac{TDoA} multilateration, the user's location can be computed.  
The remaining methods are Direct Ranging and Fingerprinting. The first one simply requires the users to wear active tags and uses different measures based on time to compute distances which are then worked by lateration techniques in order to produce the user's location. The second one works like a regular fingerprinting method except that it employs Channel Impulse Response (CIR) instead of \ac{RSSI}. This kind of fingerprinting has the possibility of being more accurate while being usable in non \ac{LOS} scenarios. On the downside it requires time synchronisation.  
One commercial example of this technology is Ubisense \cite{ubisense}, a system capable of tracking active tags equipped with batteries which have a conventional RF transceiver and a \ac{UWB} transmitter. The system requires a setup deployment of a network of Ubisensors, with fixed positions throughout the area to be covered and networked using Ethernet. Each sensor has a RF transceiver and  phased array of \ac{UWB} receivers. These sensors use a combination of \ac{TDoA} and \ac{AoA} techniques to determine the tags location, achieving an accuracy of 15 cm in a typical open environment. The system's setup can be visualised on Figure \ref{fig:ubisense}.  
  
  
  
  
\begin{figure}[H]  
\centering  
\includegraphics[width=0.7\linewidth]{2.Chapter/ubisense.png}  
\caption[Ubisense's system setup (Ref \cite{ubisense}) ]{Ubisense's system setup (Ref \cite{ubisense})}  
\label{fig:ubisense}  
\end{figure}  
  
  
  
  
\subsection{Other systems}  
\label{subsec:others}  
  
  
\begin{description}  
  
  
\item [Optical]  Indoor positioning systems are systems that use a camera as their only or main input for position estimation. In recent years these types of systems have found an increase in success due to the improvements and size reduction of the sensors, the improvements in computational capacities and the continuous development of image processing algorithms. Optical systems can be described as a moving sensor, for example a smartphone camera, and often times a set of static sensors which detect movement and which employ \ac{AoA} techniques to estimate distances. There are many different types of optical systems, one of them makes use of 3D building models. This approach removes the need for local infrastructure deployment in the building to be monitored since the usually required reference nodes are replaced by a digital reference point. As such they are highly scalable with small increases in cost.  
In general optical systems are capable of achieving high accuracy but they are vulnerable to light conditions, require \ac{LOS} propagations and are more computationally expensive than other types of systems.  
  
  
  
  
\item [\ac{FM} radio]  Is a broadcasting technology that has been incorporated for a long time on smartphones with the intent of listening to music or to the news. This technology was originally reserved for frequency modulation to convey information over a carrier wave by varying its frequency but nowadays it just refers to any radio wave in the frequency band 88-108 MHz. This analogue radio signal has amazing advantages for urban/indoor location system such as the ability to be received indoor and outdoor, it has a dense coverage in urban areas, available without installing additional transmitters, low-cost and low-power hardware with simple technology, high received signal power and there are a large number of transmitters which provides good geometry for locating. One crucial part when using \ac{FM} is that it doesn't carry any timing information which is critical in range calculation and the fact that as other radio frequency technologies, it suffers from  multipath effects and non-\ac{LOS} signals. An example of FM system was created by  et al. \cite{fm} which implemented an \ac{RSSI} fingerprint-based system using FM radios in an office environment. The system's test bed obtained 17 FM channels at each point of the fingerprint and it was capable of achieving a mean accuracy of 3 meters.  
  
  
  
  
\item [ZigBee]  Is an emerging wireless technology standard which provides solution for short and medium range communications and its specially designed for applications which demand low-power consumption and don't require large data throughput. This technology's signal range coverage can go up to 100 meters in open space, while achieving 20 to 30 meters in indoor environments. Most ZigBee-based system employ \ac{RSSI} for distance calculation and one of its most relevant disadvantage is its vulnerability to interference from a wide range of signal types using the same frequency which can disrupt radio communication. This is caused by ZigBee operating in the unlicensed ISM (industrial, scientific and medical reserved) bands. An example of a ZigBee-based system is the one created by Larrañaga et al. \cite{zigbee} which attempted to locate a mobile device in an indoor environment. Their system consisted of two phases:  
\begin{itemize} 
\item In the first phase, calibration, every ZigBee node communicated with the remaining. In this way, it was possible to work out the relationship between measured \ac{RSSI} values and geometric distances, allowing to map the environment. 
\item In the second phase, location, the mobile device communicated with the existing ZigBee nodes. This information, together with the one from the previous phase, was used to calculate the device's location. This system was capable of achieving an accuracy with an average error of 3 meters.    
\end{itemize} 
  
  
  
\item [Ultrasonic ]  Systems are employed in indoor positioning by making use of \ac{ToA} to locate targets\cite{survey3}. These kind of system make use of ultrasonic transceiver to emit and detect signals while recording times of departure and arrival of the signal. Since the signal medium traveling speed is known, it is possible to use the time difference to compute the distance between emitter and receiver.  One of the most famous projects that makes use of this technology is the cricket system which is described in Section \ref{sec:related}.  
  
  
\end{description}  
  
  
\subsection{Hybrid systems}  
\label{subsec:hybrid}  
   
Hybrid positioning systems combine several different positioning technologies to determine the location of a user or object. Hybrid systems make use of multiple technologies in an attempt to compensate for one's shortcomings through another's strengths. One example of an hybrid system is the solution presented by versus \cite{versus} which makes use of Wi-Fi, IR and RF to provide a system capable of displaying real-time locations of people or objects inside a building. By combining these three technologies their were capable of providing a system with different level of accuracy depending on the needs, room-level, bed-level (a fragment of a room) or chair-level (precise positioning).   
  
  
  
 