\section{Indoor location systems}  
\label{sec:related}  
  
  
\subsection{Active Badge}  
\label{subsec:badge}  
  
  
In 1992 the Active Badge system \cite{badge} was presented as an infrared solution capable of providing room-based position tracking. The system has been designed to make use of "active badge" beacons, which can be visualised in figure \ref{fig:badge} in the form of ID cards, a tag equipped with an \ac{IR} LED that emitted a unique code for approximately a tenth of a second every 15 seconds.  
  
  
\begin{figure}[H]  
\centering  
\includegraphics[width=0.5\linewidth]{2.Chapter/badges.png}  
\caption[Active badge's tags (Ref \cite{badgefig}) ]{Active badge's tags (Ref \cite{badgefig}) }  
\label{fig:badge}  
\end{figure}  
  
  
The decision to use \ac{IR} was caused by the small and cheap characteristics of the emitters and detectors\cite{badge2}. By being capable of operating within a 6 meter range, \ac{IR} signals aren’t capable of traveling through walls. The signal frequency has two major effects on the system: Firstly it impacts the tag's energy consumption, where a small frequency allows for long periods of work on a single battery; Secondly it impacts on user detection accuracy. For the used signal duration and frequency there is a chance of 1/150 for two signals to collide, which leads to a good probability that for a small number of beacons, all will be detected. One downside of such a small frequency signal is that the location of a badge can only be known, at best, with a 15 second granularity.  
  
The position of a user is obtained through the implementation of a network of sensors which act as receivers, continuously listening to badge transmissions. Upon collecting a transmission, the obtained information is sent to the master station. The master station is responsible for polling all the sensors on the network, storing sighted badges into a database where its associated time, position and ID are stored, data processing and data display. The accuracy of the system is room-based by making use of \ac{CoO} and the properties of \ac{IR}. A beacon in each room would make it so that each beacon is capable of detecting any badges in its room \cite{badge1}.  
  
  
\subsection{Active Bat}  
\label{subsec:bat}  
  
  
In 2001 the Active bat system \cite{bat}was introduced. It is a system capable of tracking various object, each tagged by attaching small wireless transmitters called bats. The system's architecture is composed of small devices named bats, which are to be carried by the objects or people to be tracked, a network of \acf{US} receiver units and several base stations\cite{bat1}. The receiver network and a deployed bat can be seen on figure \ref{fig:bat};  
  
  
\begin{figure}[H]  
\centering  
\includegraphics[width=0.5\linewidth]{2.Chapter/batarch.png}  
\caption[Active bat network and device(Ref \cite{batfig}) ]{Active bat network and device (Ref \cite{batfig}) }  
\label{fig:bat}  
\end{figure}  
  
  
A bat, which can be seen in figure \ref{fig:bat} being carried by the user, consisted of a radio transceiver, controlling logic and a ultrasonic transducer, each having an associated globally unique ID. A Base station periodically transmits a radio message containing a single unique ID, making it so that the ID's associated bat emits a short pulse of ultrasound. From this point on, the receivers monitor for the expected \ac{US} signal while recording the time spent waiting in order to obtain the signal's \ac{ToA}. With a known speed of sound in air, which can be estimated from the ambient temperature, the \ac{ToA} can be converted into bat-receiver distance.  
  
  
The mobile target's position can be obtained through multilateration in three dimensional space, for as long as three or more non-collinear receivers' distances are known. This method's accuracy is highly dependent on the distance measurement's accuracy. Distance measurement is affected by signal reflections on objects present in the environment, a problem that was correct by the use of a statistical outlier rejection algorithm. One other issue is the reverberations of the initial signal, which are required to die out before initiating another distance measurement in order to ensure that the incoming \ac{US} signals are from the correct bat. As such the measurement process is divided into time-slots, each being capable of locating one and only one bat.  
  
  
The latest version of the bat included a sensitive motion detector that allowed it to tell the base stations whether it was moving or stationary. Since the base station doesn't require to repeatedly determine the location of a stationary object, the system places these bats into a low-power sleep state which is only removed once the bat starts moving. This implementation allowed for extra power savings while freeing up location-update opportunities for other bats \cite{bat2}.  
  
  
\subsection{Radar}  
\label{subsec:radar}  
  
  
In 2001 the RADAR system was introduced as the first Wi-Fi signal-strength based indoor positioning system \cite{radar}. The system is Radio Frequency-based and its capable of locating and tracking users inside buildings. Radar makes use of signal-strength information obtained through a fingerprint method, presented in subsection \ref{subsubsec:rssi}, to triangulate the user's coordinates. The system's functions in two phases: the data collection phase, where the data is gathered in order to later construct and validate models for signal propagation, which are to be used in the real-time phase to infer user's location. In the offline phase, the type of data collected is the signal strength utilising the methodology already described for the fingerprint method.  
  
  
Radar's architecture involved a mobile device, base stations and a central station. The mobile device is solely responsible for sending signals to the base stations. During the process of developing the system, the device was a Windows-based mobile host capable of broadcasting packets. The base stations act as gateways, in a way that it listens for the user's signals and consequently forwards its information to the central station. The central station is in charge of storing the whole fingerprint information in the offline phase and calculating the user's location during the online phase.  
  
  
Data processing involved computing the mean, the standard deviation and the median of \ac{RSSI} for each of the used base stations (three in total) and each combination of x,y and direction. In addition, a building layout information was created which included room and base station's coordinates and the number of walls that obstructed the direct line between the base stations and each of the positions where data was collected. With all this information an accurate signal propagation model was built.  
  
  
The basic approach used to obtain a user's location was triangulation, which given a set of \ac{RSSI} measurements at each base station, the user's location is predicted to be the one that best matches the observed data. In addition to this basic strategy, two others approaches were analysed: empirical and signal propagation methods.   
  
  
For the first method many variations on the data were studied such as: The number of best matching values used ( K-nearest neighbours (KNN)). The results showed that the benefits of averaging between multiple neighbours isn't very relevant, even for small values of k. In general the empirical method was capable of estimating the user's location with high accuracy, obtaining a median error distance between 2 and 3 meters. Its main drawback being the required effort for building the data set for each physical area of interest. Another issue is the requirement to remake the data collection phase whenever a base stations is moved or there are heavy changes in the environment.  
  
  
The signal propagation model comes as an alternative to the empirical method for constructing the fingerprint. This method makes use of a propagation model for the signal to generate a set of theoretically-computed signal strength data, similar to the one physically collected. The performance of this method is correlated to how well the used model is capable of correctly describing the signal. This model provides a more reasonable way of obtaining data, since it doesn't require detailed and costly measurements. When compared to the empirical method, it was capable of achieving a mean error distance of 4.5 meters. Although it isn't as accurate, it can be considerate as a solution when analysing its benefits.  
  
  
\subsection{Cricket}  
\label{subsec:cricket}  
  
  
The Cricket system\cite{cricket} was developed by the Massachusetts Institute of Technology (MIT) in 2005 and managed to tackle some of the problems existent in the previously mentioned systems. The system makes use of nodes, small hardware platforms, consisting of a RF transceiver, a microcontroller and hardware capable of generating and receiving ultrasonic signals. There are two types of nodes: beacons, which are fixed reference points attached to the ceiling or walls of the building, and receivers, called listeners, which are attached to the objects that need to be tracked. Each beacon periodically transmits a \ac{RF} signal message containing beacon specific information, such as the beacon's unique ID, its coordinates and the physical space associated to the beacon. Whenever a \ac{RF} signal is transmitted, an ultrasonic pulse, which doesn't contain any data, is also emitted thus enabling listeners to measure their distance to the beacons by using the time difference of arrival times of the RF and ultrasonic signals. Each listener utilises the RF signal's beacon information alongside the obtained distances to beacons to compute their space position and orientation.   
  
  
When a beacon is deployed it doesn't know its exact position, only a human-readable string which describes its location. In order to compute the recently deployed beacon's position a listener is attached to a roaming device in order to collect distances from the beacons to itself. These distances are used to compute inter-beacon distances, which, when in high enough number, are capable of uniquely define how beacons are located in respect to each other. With this information it is then possible to obtain the beacon's coordinates.  
  
  
Cricket's architecture is completely decentralized, since it doesn't require a central node. The used nodes are responsible for communicating with the beacons and calculating their own location, while the beacons are responsible for finding their positions and transmitting their information to the listeners.  
  
  
The utilised method for computing distances doesn't require listeners to actively transmit messages which permits cricket to perform well independently of the number of users. This active-beacon passive-listener architecture makes it so that the position of the user isn't tracked by the system thus solving the user privacy issues that were present in the remaining projects.  
  
  
\begin{figure}[H]  
\centering  
\includegraphics[width=0.5\linewidth]{2.Chapter/cricket-dist.png}  
\caption[Cricket's \ac{TDoA} representation]{Cricket's \ac{TDoA} representation}  
\label{fig:cricket-tdoa}  
\end{figure}  
  
  
Distance measurement is computed through \ac{TDoA} using both \ac{RF} and \ac{US} signals and it can be visualised in figure \ref{fig:cricket-tdoa}. Since the velocity of the \ac{RF} signal is much higher than that of the \ac{US} signal, the \ac{US} signal lags behind when compared to the other. As such, whenever a listener receives a \ac{RF} signal, it measures the time it takes until the \ac{US} signal arrives, denominated $\delta{T}$. With knowledge of the speeds of sound and light, the distance between beacon and listener can be obtained from:  
  
  
\begin{equation}  
\delta t = \frac{d}{v_{US}} - \frac{d}{v_{RF}}  
\end{equation}  
  
  
This approach to distance computation is vulnerable to a certain amount of factors such as: Environmental factors since the velocity of sound depends on factors such as temperature, humidity and atmospheric pressure; Lack of \ac{LOS} since in these scenarios there is no \ac{LOS} between the beacon and the listeners, the \ac{US} signal may reach the listener after it has reflected on a surface. Reflection and refraction cause the signal to travel a longer distance than the direct path; Errors in detecting \ac{US} due to the threshold-based approach to detect the signal; \ac{TDoA} associated errors, which are associated to errors from time measurement and errors from detecting the \ac{RF} signal.  
  
  
In terms of performance, Cricket was capable of a distance measurement accuracy of 4-5 cm within a 80 degree cone from a given beacon, position accuracy of 10-12 cm and an orientation accuracy of 3º - 5º.  
  
  
\subsection{BLE systems}  
\label{subsec:ble}  
When looking at what's possible to achieve using the BLE technology there is the example of Apple's creation iBeacon \cite{ibeacon} which was presented in 2013 with the purpose of implementing proximity sensing systems. The devices are implemented as broadcasters, making it so that its objective is to send information to nearby compatible receivers. Some examples of application are to track customers or trigger location-based actions on devices such as push notifications or checking in on social media, with practical cases such as the usage of iBeacons by McDonalds to offer special offers to their customers in their fast-food stores. An indoor location system using this technology was presented by Jingjing Yang et al \cite{ibeacon1}, where these devices were used to indicate a patient of his whereabouts through the proximity sensing properties and this information was later transferred over to a server in order to give clients a variety of different services, from patient counting, to nearby department's information and offer indoor guidance to the nearest available bed.   
  
  
When using BLE for indoor location, the metric used to calculate distances is the RSSI. This metric within the context of Bluetooth brings to the surface several issues such as the fact that RSSI as a metric is very accurate only when the target is within a meter of the beacon, since the value decreases with the inverse of the square of the distance to the beacon. As such when developing solutions for indoor location that require systems with high accuracy capable of tracking moving objects, the usage of RSSI can't be utilised without further work. Faragher et al \cite{bleacc} tackled one of the techniques used to improve BLE system's accuracy, fingerprinting, by verifying the effects caused by the device deployment density within the required location. This experiment also puts into evidence one of the downsides of the Bluetooth technology: low scalability; requires higher density in order to increase accuracy; due to their low range, increasing coverage leads to increased costs.  
  
  
  
  
Zonith \cite{zonith} introduced a Bluetooth based location system with the objective of tracking the position of workers in dangerous environments. Any device registered in the zonith implemented network would be continuously tracked and accounted for in each of the system's functionalities such as, sounding an alarm whenever a lone worker doesn't move or respond within a time interval (Lone worker protection) or providing a quick a precise location of any worker that has requested for help. This system's installation requires planning of the best locations to place the beacons and number of beacons required, in order to be able to provide enough coverage and assure the system's required quality.  
  
 